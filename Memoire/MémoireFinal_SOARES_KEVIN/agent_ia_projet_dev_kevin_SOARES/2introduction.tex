\chapter{Introduction}

L’essor des modèles de langage de grande taille (\textbf{Large Language Models} ou LLMs) a bouleversé la façon dont nous interagissons avec le texte : un simple prompt suffit désormais à générer une page de prose, un résumé d’article ou, plus surprenant encore, un extrait de code compilable.  Dans la foulée, une nouvelle génération d’agents IA, capables de raisonner sur plusieurs tours, d’appeler des outils externes et de \textbf{collaborer entre eux}, prétend couvrir l’ensemble du cycle logiciel : rédaction des spécifications, écriture des tests, production du code, déploiement continu.  Face à ces annonces, une question surgit : \textbf{Une équipe de développement peut-elle, à terme, être remplacée entièrement par des agents IA ?}

Ce mémoire propose de répondre à cette interrogation en adoptant une démarche bibliographique structurée.  Nous passons d’abord en revue les principes des LLMs et les architectures multi-agents, puis nous confrontons leurs performances à des benchmarks de génération de code.  Nous analysons ensuite trois prototypes représentatifs : \textit{CodePori}, \textit{Agent-Driven Automatic Software Improvement} et un pipeline Auto-DevOps multi-agents, qui illustrent le potentiel, mais aussi les limites, d’équipes 100 \% IA (ou presque). Enfin, nous synthétisons ces observations via une matrice SWOT et nous esquissons trois scénarios d’évolution, de la programmation assistée d’aujourd’hui à l’autonomie (quasi) complète de demain.

La contribution essentielle de ce travail est double.  D’une part, il offre une \textbf{cartographie des tâches} qu’une équipe de développement doit couvrir et un état de la couverture actuelle par les agents IA.  D’autre part, il identifie les \textbf{conditions critiques} (alignement, sécurité, gouvernance) sans lesquelles le remplacement total demeurerait théorique.  L’objectif n’est pas de trancher définitivement, mais de fournir au lecteur une grille d’analyse argumentée pour évaluer la faisabilité, les risques et les opportunités d’un tel bouleversement.

Le plan suit une logique progressive : après l’état de l’art \ref{chapitre:etatArtLLM} et l’étude de la dynamique équipe humaine/agents \ref{chapitre:humainAgent}, nous passons aux cas d’usage concrets \ref{chapitre:etudeCas}, puis à une discussion de faisabilité et de perspectives \ref{chapitre:faisabilite}.  La conclusion \ref{conclusion} revient sur les points clés et ouvre sur les recherches futures, qu’elles soient techniques, organisationnelles ou éthiques.

En somme, ce mémoire vise à montrer pourquoi le sujet est crucial, comment nous l’avons abordé, et en quoi nos résultats peuvent éclairer décideurs, chercheurs et praticiens.  Le lecteur n’a pas besoin d’être expert en IA : il trouvera ici les repères nécessaires pour comprendre les promesses, les limites et les implications d’une \textbf{éventuelle équipe de développement sans développeurs}.
